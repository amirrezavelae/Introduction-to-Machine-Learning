\documentclass[12pt]{article}
\usepackage{blindtext}
\usepackage[en,bordered]{uni-style}
\usepackage{uni-math}
\usepackage{physics}
\usepackage{amssymb}
\usepackage{capt-of}
\DeclareMathOperator*{\argmax}{arg\,max}
\DeclareMathOperator*{\argmin}{arg\,min}
\title{Intruduction to Machine Learning}
\prof{Dr \,S.Amini}
\subject{Homework 3}
\info{
    \begin{tabular}{lr}
        Amirreza Velae & 400102222\\
        github    & \href{https://github.com/amirrezavelae}{repository}\\
    \end{tabular}
    }
    \date{\today}
    % \usepackage{xepersian}
    % \settextfont{Yas}
    \usepackage{uni-code}
    
\begin{document}
\maketitlepage
\maketitlestart

\section{Design simple neural network}
Design a simple neural network with one hidden-layer that implements the following function:
\begin{gather*}
    (A\vee \bar{B}) \oplus (\bar{C} \vee \bar{D})
\end{gather*}
Draw the network and determine all its weights.
\begin{qsolve}
    \begin{gather*}
        (A\vee \bar{B}) \oplus (\bar{C} \vee \bar{D}) = ((A\vee \bar{B}) \vee (C \wedge D)) \vee ((\bar{A} \wedge B) \vee (\bar{C} \vee \bar{D}))\\
        = ((A\vee \bar{B} \vee C) \wedge (A\vee \bar{B} \vee D)) \vee ((\bar{A} \vee \bar{C} \vee \bar{D}) \wedge (B \vee \bar{C} \vee \bar{D}))
    \end{gather*}
\end{qsolve}




















\makeendpage
\end{document}